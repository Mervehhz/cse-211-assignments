\documentclass[a4 paper]{article}
\usepackage[inner=2.0cm,outer=2.0cm,top=2.5cm,bottom=2.5cm]{geometry}
\usepackage{setspace}
\usepackage[rgb]{xcolor}
\usepackage{verbatim}
\usepackage{subcaption}
\usepackage{amsgen,amsmath,amstext,amsbsy,amsopn,tikz,amssymb}
\usepackage[colorlinks=true, urlcolor=blue,  linkcolor=blue, citecolor=blue]{hyperref}
\usepackage[colorinlistoftodos]{todonotes}
\usepackage{rotating}
\usepackage{booktabs}
\newcommand{\ra}[1]{\renewcommand{\arraystretch}{#1}}

\newtheorem{thm}{Theorem}[section]
\newtheorem{prop}[thm]{Proposition}
\newtheorem{lem}[thm]{Lemma}
\newtheorem{cor}[thm]{Corollary}
\newtheorem{defn}[thm]{Definition}
\newtheorem{rem}[thm]{Remark}
\numberwithin{equation}{section}

\newcommand{\homework}[6]{
   \pagestyle{myheadings}
   \thispagestyle{plain}
   \newpage
   \setcounter{page}{1}
   \noindent
   \begin{center}
   \framebox{
      \vbox{\vspace{2mm}
    \hbox to 6.28in { {\bf CSE 211:~Discrete Mathematics \hfill {\small (#2)}} }
       \vspace{6mm}
       \hbox to 6.28in { {\Large \hfill #1  \hfill} }
       \vspace{6mm}
       \hbox to 6.28in { {\it Instructor: {\rm #3} \hfill Name: {\rm #5} \hfill Student Id: {\rm #6}} \hfill}
       \hbox to 6.28in { {\it Assistant: #4  \hfill #6}}
      \vspace{2mm}}
   }
   \end{center}
   \markboth{#5 -- #1}{#5 -- #1}
   \vspace*{4mm}
}

\newcommand{\problem}[2]{~\\\fbox{\textbf{Problem #1}}\hfill (#2 points)\newline\newline}
\newcommand{\subproblem}[1]{~\newline\textbf{(#1)}}
\newcommand{\D}{\mathcal{D}}
\newcommand{\Hy}{\mathcal{H}}
\newcommand{\VS}{\textrm{VS}}
\newcommand{\solution}{~\newline\textbf{\textit{(Solution)}} }

\newcommand{\bbF}{\mathbb{F}}
\newcommand{\bbX}{\mathbb{X}}
\newcommand{\bI}{\mathbf{I}}
\newcommand{\bX}{\mathbf{X}}
\newcommand{\bY}{\mathbf{Y}}
\newcommand{\bepsilon}{\boldsymbol{\epsilon}}
\newcommand{\balpha}{\boldsymbol{\alpha}}
\newcommand{\bbeta}{\boldsymbol{\beta}}
\newcommand{\0}{\mathbf{0}}


\begin{document}
\homework{Homework \#4}{Due: 17/01/21}{Dr. Zafeirakis Zafeirakopoulos}{Gizem S\"ung\"u}{}{}
\textbf{Course Policy}: Read all the instructions below carefully before you start working on the assignment, and before you make a submission.
\begin{itemize}
	\item It is not a group homework. Do not share your answers to anyone in any circumstance. Any cheating means at least -100 for both sides. 
	\item Do not take any information from Internet.
	\item No late homework will be accepted. 
	\item For any questions about the homework, send an email to gizemsungu@gtu.edu.tr
	\item The homeworks (both latex and pdf files in a zip file) will be
	submitted into the course page of Moodle.
	\item The latex, pdf and zip files of the homeworks should be saved as
	"Name\_Surname\_StudentId".$\{$tex, pdf, zip$\}$.
	\item If the answers of the homeworks have only calculations without any formula or any explanation -when needed- will get zero.
	\item Writing the homeworks on Latex is strongly suggested. However, hand-written paper is still accepted $\textbf{IFF}$ hand writing of the student is clear and understandable to read, and the paper is well-organized. Otherwise, the assistant cannot grade the student's homework.
\end{itemize}

\problem{1}{15+15=30}
Consider the nonhomogeneous linear recurrence relation $a_n$ = 3$a_{n-1}$ + $2^n$ .\\
\subproblem{a} Show that whether $a_n$ = $-2^{n+1}$ is a solution of the given recurrence relation or not. Show your work step by step.
\solution
\newline
\newline
if $a_n$ = $-2^{n+1}$ ,then what is $a_{n-1}$ ?
\newline
$a_{n-1}$ = $-2^{n-1+1}$ = $-2^n$
\newline
if we plug the $a_{n-1}$ into $a_n$ = 3$a_{n-1}$ + $2^n$ ;
\newline
$a_n$ = 3 x $-2^n$ + $2^n$
\newline
$a_n$ = $2^n$ (-3 + 1) = $2^n$ x (-2)
\newline
$a_n$ = $-2^{n+1}$
\newpage
\subproblem{b} Find the solution with $a_0$ = 1.
\solution
\newline
$a_n$ = 3$a_{n-1}$ + $2^n$  and $a_0$ = 1.
\newline
non-homogeneous linear recurrences have two parts. One of these homo part, other is particular part.
And this is represented as $a_n$ = $a_n^h$ + $a_n^p$
\newline
particular solution is same as in part a.
That is $a_n^p$ =  $-2^{n+1}$
\newline
$a_n$ - 3$a_{n-1}$ = 0
\newline
if we divide by $a_{n-1}$, then the characteristhic equation is found.
That is ;
\newline
r - 3 = 0,  r = 3.
\newline
$a_n^h$ = A x $3^n$
\newline
$a_n$ = A x $3^n$ + $-2^{n+1}$
\newline
for $a_0$ = 1
\newline
A x $3^0$ $-2^{0+1}$
\newline
A-2 = 1 , A = 3
\newline
The equation is :  $3^{n+1} - 2^{n+1}$
\newline
\problem{2}{35}
Solve the recurrence relation f(n) = 4f(n-1) - 4f(n-2) + $n^2$ for f(0) = 2 and f(1) = 5. 
\solution
\newline
non-homogenous recurrence releation has two parts: homogenous part and particular part.
\newline
\newline
homogeneous part is written on one side and particular part on the other side of the equality.
\newline
f(n) - 4f(n-1) + 4f(n-2) = $n^2$
\newline
\newline
$f(n)^{(g)}$ = $f(n)^{(h)}$ + $f(n)^{(p)}$
\newline 
\newline
finding characteristhic equation with homogenous part : 
\newline
\newline
$f(n)^{(h)}$ = f(n) - 4f(n-1) + 4f(n-2)
\newline
\newline
if we divide each term by f(n-2) ;
\newline
characteristic equation ;
\newline
$r^2$ -4r + 4 = 0
\newline
=\> (r-2)(r-2) = 0      => r=2
\newline
\newline
$f(n)^{(h)}$ = p . $2^n$ + q.n.$2^n$
\newline
\newline
solving particular part :
\newline
\newline
 $f(n)^{(p)}$ = A.$n^2$ + B.n + C
\newline
\newline
Putting f(n) into the original equation ;
\newline
\newline
A.$n^2$ + B.n + C = 4[A.$(n-1)^2$ + B.(n-1) + C] -4[A.$(n-2)^2$ + B.(n-2) + C] + $n^2$
\newline
\newline
A.$n^2$ + B.n + C = 4.[A.($n^2$ -2.n +1) + B.n -B + C] -4.[A.($n^2$ -4.n + 4) + B.n -2.B + C] + $n^2$
\newline
\newline
A.$n^2$ + B.n + C = 4A$n^2$ -8An + 4A +4Bn -4B + 4C - 4A$n^2$ + 16An -16A - 4Bn + 8B -4C + $n^2$
\newline
\newline
A$n^2$ + B.n + C = $n^2$ + 8An -12A + 4B
\newline
\newline
variables of the same degree must be equal to the same coefficient
\newline
\newline
A = 1, B = 8, C = -12.1 + 4.8 = 20
\newline
\newline
$f(n)^{(p)}$ = $n^2$ + 8n + 20
\newline
\newline
f(n) = p.$2^n$ + q.n.$2^n$ + $n^2$ + 8n + 20
\newline
\newline
f(0) = 2 , f(1) = 5
\newline
\newline
f(0) = p + 20 = 2
\newline
p = -18
\newline
\newline
f(1) = 2.(-18) + 2.q + 1 + 8 + 20 = 5
\newline
\newline
q = 6
\newline
\newline
f(n) = -18.$2^n$ + 6.n.$2^n$ + $n^2$ + 8n + 20

\newpage
\problem{3}{20+15 = 35}
Consider the linear homogeneous recurrence relation $a_n$ = 2$a_{n-1}$ - 2$a_{n-2}$.
\subproblem{a} Find the characteristic roots of the recurrence relation.
\solution
\newline
\newline
$a_n$ - 2$a_{n-1}$ + 2$a_{n-2}$ = 0
\newline
characteristhic equation :
\newline
$r^2$ - 2r + 2 = 0
\newline
delta = $b^2$ -4ac = $(-2)^2$ -4*1*2 = -4
\newline
\newline
first root = (-b + $\sqrt{delta}$)/(2*a) = (-(-2) + 2i)/2 = 1+i
\newline
second root = (-b - $\sqrt{delta}$)/(2*a) = (-(-2) - 2i)/2 = 1-i
\newline
\subproblem{b} Find the solution of the recurrence relation with $a_0$ = 1 and $a_1$ = 2.
\solution
\newline
\newline
$a_n$ = A $(1+i)^n$ + B $(1-i)^n$
\newline
$a_0$ = 1 = A + B (first equation)
\newline
$a_1$ = 2 = A x (1+i) + B x (1-i)
\newline
A * i - B * i = 1
\newline
A-B = 1/i (second equation)
\newline
if we find A and B by using first and second equation,
\newline
A = (i+1) / 2i
\newline
B = (i-1) / 2i
\newline
\newline
$a_n$ = ((i+1) / 2i) x $(1+i)^n$ + ((i-1) / 2i) x $(1-i)^n$

\end{document} 

